\chapter{Planteamiento del Proyecto}
Es la especificación inicial del proyecto la cual incluye la problemática que se trata de solucionar por medio de la investigación y, para la tesis, es en sí la elección del tema que servirá de base para elaborarla mediante una proposición concreta indicando el contexto preliminar del problema, definición del problema, objetivos, justificación/relevancia.

\section{Planteamiento del problema}
Descripción del contexto y definición del problema el cual deberá cumplir una serie de condiciones que de alguna forma justifiquen el esfuerzo necesario para resolverlo. Entre ellas: originalidad, trascendencia, actualidad, relevancia y la posibilidad de permitir el uso de lo aprendido a lo largo de la carrera.

\section{Objetivos del Proyecto}
%% La autoindentación dentro de subsecciones no es posible por lo que estas se manejan como ambientes separados.
\begin{subseccion}{Objetivo General}
Descripción de la finalidad principal que persigue el trabajo, el motivo que le dará vigencia. 
\end{subseccion}

\begin{subseccion}{Objetivos Específicos}
Señalan las actividades que se deben cumplir para avanzar con el proyecto y lo que se pretende lograr en cada una de las etapas de ella, por ende, la suma de los resultados de cada uno de los objetivos específicos permitirán alcanzar el propósito integral del objetivo general. Especifica los logros concatenados que se pretende conseguir.
Para la formulación de los objetivos considere lo siguiente:

\begin{itemize}
    \item Deben estar dirigidos a los elementos básicos del problema
    \item Deben ser medibles y observables
    \item Deben ser claros y precisos
    \item Su formulación debe involucrar resultados concretos
    \item El alcance de los objetivos debe estar dentro de las posibilidades del investigador
    \item Deben ser expresados en verbos en infinitivos
\end{itemize}
\end{subseccion}

\section{Justificación}
Expone de manera lógica aspectos como:

\begin{itemize}
    \item Importancia de la investigación.
    \item Conveniencia del estudio.
    \item Aportes/beneficios al dominio.
    \item Implicación práctica.
    \item Utilidad metodológica.
\end{itemize}

\section{Viabilidad}

Describir claramente las viabilidades económicas, técnicas y operativas del proyecto

\section{Limitaciones}

Se describen las posibles implicancias y/o dificultades que limiten el correcto desarrollo de la tesis, afectando el alcance, viabilidad, objetivos, entre otros aspectos.

\section{Descripción de la Propuesta}

Describir claramente la propuesta del trabajo de investigación de tal forma que sea congruente con lo indicado en los objetivos