\documentclass[12]{plan_tesis}

\usepackage[spanish]{babel}
\usepackage[utf8]{inputenc}
\usepackage[a4paper, total={15.5cm, 24.7cm}]{geometry}

\usepackage{appendix}
\usepackage{graphicx}
\graphicspath{ {Imagenes/} }

\usepackage{xcolor}

\title{[TITULO DEL  INFORME DE INVESTIGACIÓN]}
\author{WardoPo}
\date{March 2021}

\begin{document}

\input{portada}

\begin{resumen}
El resumen posee un conjunto de elementos los cuales dan una visión clara del trabajo descrito en no más de 120 palabras en un solo párrafo.  

La forma de redacción es en pasado impersonal, por ejemplo: una vez que recabamos la información (esta forma es incorrecta), en vez de ello se debe decir: una vez que se recabó la información (estas la forma correcta).  
\end{resumen}

\chapter{Planteamiento del Proyecto}
Es la especificación inicial del proyecto la cual incluye la problemática que se trata de solucionar por medio de la investigación y, para la tesis, es en sí la elección del tema que servirá de base para elaborarla mediante una proposición concreta indicando el contexto preliminar del problema, definición del problema, objetivos, justificación/relevancia.

\section{Planteamiento del problema}
Descripción del contexto y definición del problema el cual deberá cumplir una serie de condiciones que de alguna forma justifiquen el esfuerzo necesario para resolverlo. Entre ellas: originalidad, trascendencia, actualidad, relevancia y la posibilidad de permitir el uso de lo aprendido a lo largo de la carrera.

\section{Objetivos del Proyecto}
%% La autoindentación dentro de subsecciones no es posible por lo que estas se manejan como ambientes separados.
\begin{subseccion}{Objetivo General}
Descripción de la finalidad principal que persigue el trabajo, el motivo que le dará vigencia. 
\end{subseccion}

\begin{subseccion}{Objetivos Específicos}
Señalan las actividades que se deben cumplir para avanzar con el proyecto y lo que se pretende lograr en cada una de las etapas de ella, por ende, la suma de los resultados de cada uno de los objetivos específicos permitirán alcanzar el propósito integral del objetivo general. Especifica los logros concatenados que se pretende conseguir.
Para la formulación de los objetivos considere lo siguiente:

\begin{itemize}
    \item Deben estar dirigidos a los elementos básicos del problema
    \item Deben ser medibles y observables
    \item Deben ser claros y precisos
    \item Su formulación debe involucrar resultados concretos
    \item El alcance de los objetivos debe estar dentro de las posibilidades del investigador
    \item Deben ser expresados en verbos en infinitivos
\end{itemize}
\end{subseccion}

\section{Justificación}
Expone de manera lógica aspectos como:

\begin{itemize}
    \item Importancia de la investigación.
    \item Conveniencia del estudio.
    \item Aportes/beneficios al dominio.
    \item Implicación práctica.
    \item Utilidad metodológica.
\end{itemize}

\section{Viabilidad}

Describir claramente las viabilidades económicas, técnicas y operativas del proyecto

\section{Limitaciones}

Se describen las posibles implicancias y/o dificultades que limiten el correcto desarrollo de la tesis, afectando el alcance, viabilidad, objetivos, entre otros aspectos.

\section{Descripción de la Propuesta}

Describir claramente la propuesta del trabajo de investigación de tal forma que sea congruente con lo indicado en los objetivos
\chapter{Revisión y Fundamentación Teórica}

\section{Estado del Arte}
En este apartado, considerado también como marco de referencia o estado del arte, se deberá analizar todo aquello que se ha escrito acerca del objeto de estudio: ¿qué se sabe del tema? ¿qué estudios se han hecho en relación a él? ¿desde qué perspectivas se ha abordado?\newline

Se debe evitar ahondar en teorías que sólo planteen un solo aspecto del fenómeno.\newline

Las funciones de los antecedentes son:

\begin{itemize}
    \item Delimitar el área de investigación
    \item Hacer un compendio de estudios realizados relacionados al tema de investigación
    \item Ayudar a prevenir errores que se han cometido en otros estudios
    \item Orientar sobre cómo habrá de llevarse a cabo el estudio
    \item Proveer un marco de referencia para interpretar los resultados del estudio
\end{itemize}

Esto implica realizar una revisión crítica de la literatura correspondiente, pertinente y actualizada. Al final, es importante fijar una determinada postura ante el fenómeno en cuestión.

\section{Fundamentos teóricos}

Este apartado expone la sustentación teórica del problema de investigación u objeto de estudio, realizando un compendio de conocimientos existentes en el área que se va a investigar, expresando proposiciones teóricas.

\chapter{Plan de Gestión del Proyecto}

\section{Alcance}

Precisar los procesos, organismos, áreas, funciones necesarias que se requieren para garantizar con éxito el proyecto. Básicamente es definir qué se incluye y que no se incluye en el proyecto.

\section{Metodología de desarrollo}

Indicar el tipo de metodología de desarrollo que se utilizará para la implementación de la propuesta y explicar de forma general los procedimientos, procesos y/o actividades de los que consta la metodología.

\section{Plan de Trabajo y Cronograma}

Especificar los tiempos a emplear en cada etapa del proceso del trabajo para ordenar las actividades y programarlas. Las actividades deben ir numeradas y debe presentarse a través de un Diagrama de Gantt. Este cronograma sirve para controlar y monitorizar el desarrollo del trabajo.

\section{Plan de Recursos}

Especificar los recursos materiales, de personas o equipos para el proyecto. \cite{einstein}

\bibliographystyle{IEEEtran}
\bibliography{bibliografia}

\input{Anexos/Ejemplo}

\end{document}
