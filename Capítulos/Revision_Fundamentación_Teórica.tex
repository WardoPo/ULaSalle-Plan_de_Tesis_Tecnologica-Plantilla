\chapter{Revisión y Fundamentación Teórica}

\section{Estado del Arte}
En este apartado, considerado también como marco de referencia o estado del arte, se deberá analizar todo aquello que se ha escrito acerca del objeto de estudio: ¿qué se sabe del tema? ¿qué estudios se han hecho en relación a él? ¿desde qué perspectivas se ha abordado?\newline

Se debe evitar ahondar en teorías que sólo planteen un solo aspecto del fenómeno.\newline

Las funciones de los antecedentes son:

\begin{itemize}
    \item Delimitar el área de investigación
    \item Hacer un compendio de estudios realizados relacionados al tema de investigación
    \item Ayudar a prevenir errores que se han cometido en otros estudios
    \item Orientar sobre cómo habrá de llevarse a cabo el estudio
    \item Proveer un marco de referencia para interpretar los resultados del estudio
\end{itemize}

Esto implica realizar una revisión crítica de la literatura correspondiente, pertinente y actualizada. Al final, es importante fijar una determinada postura ante el fenómeno en cuestión.

\section{Fundamentos teóricos}

Este apartado expone la sustentación teórica del problema de investigación u objeto de estudio, realizando un compendio de conocimientos existentes en el área que se va a investigar, expresando proposiciones teóricas.
